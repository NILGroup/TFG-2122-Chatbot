\chapter*{Resumen}

Este trabajo se centra en el desarrollo de un Chatbot que recopile y almacene los recuerdos de una persona con demencia. El objetivo es facilitar la tarea de los terapeutas con ese paso previo a la construcción de la historia de vida de un paciente. Se trata de una herramienta de apoyo de cara a las terapias basadas en reminiscencia. 

%De forma complementaria al bot conversacional y pra darle una versi, se crea una aplicación web dirigida tanto a terapeutas como a pacientes cuyo propósito es 
 
El chatbot se ha programado en base a tres componentes principales. El primero es un analizador de sentimientos que detecta si un texto tiene connotaciones negativas o positivas. En segundo lugar se encuentra el clasificador de recuerdos en etapas de vida que indica si el texto se corresponde con un recuerdo de la infancia, de la juventud, de la etapa adulta o de la vejez. El último componente es un módulo que elige cuál es la mejor siguiente pregunta a formular, la que más relación tenga con la respuesta que haya dado el usuario. 

Para que la interacción con el Chatbot sea atractiva, visual y fácil, se ha creado una aplicación web complementaria dirigida tanto a terapeutas como a pacientes. Aunque el foco principal de la aplicación sea el bot conversacional, también se ofrecen otras funcionalidades como la posibilidad de revisar los recuerdos ya almacenados o la de crear nuevas terapias.

El código desarrollado en el proyecto puede
encontrarse en \url{https://github.com/NILGroup/TFG-2122-Chatbot}. 


\section*{Palabras clave}
   
Chatbot, aplicación web, demencia, recuerdo, terapia, historia de vida

   


