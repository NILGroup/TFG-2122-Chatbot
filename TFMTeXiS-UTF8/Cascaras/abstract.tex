\chapter*{Abstract}

This project focuses on the development of a Chatbot that collects and stores the memories of people who suffer dementia. The main goal is to facilitate the therapist's job by helping with the steps prior to shapping a patient's life story. It's a support tool for reminiscence-based therapies.

The Chatbot has been programmed considering three main components. The first one is a sentiment analyser that detects whether the text contains negative or positive connotations. Secondly, we have a memory classifier that indicates whether the text corresponds to a memory from the childhood, youth, adulthood or old age. The last component is a module that chooses the next best question to be asked, the one that is most related to the answer given by the user.

To make the interaction with the Chatbot attractive, visual and easy, a complementary web application has been created for both therapists and patients. Although the main focus of the app is the conversational bot, other functionalities are also offered, such as the possibility of reviewing the already stored memories or creating new therapies.

The code developed throughout the project can be found at \url{https://github.com/NILGroup/TFG-2122-Chatbot}.


\section*{Keywords}

\noindent Chatbot, web application, dementia, memory, therapy, life stories



