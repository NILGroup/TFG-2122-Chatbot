\chapter{Introducción}
\label{cap:introduccion}

\chapterquote{Frase célebre dicha por alguien inteligente}{Autor}

Introducción temporal

Las personas con Alzheimer u otros tipos de demencia pueden beneficiarse del uso de la llamada terapia basada en reminiscencia, que se basa en la construcción de un libro de vida del paciente que recopila recuerdos positivos de su vida que se pueden utilizar posteriormente par ejercitar su memoria y retrasar el deterioro, además de permitir aumentar el bienestar de los pacientes.

En el presente proyecto se propone el desarrollo de un chatbot que permita recopilar y estructurar esta información para ayudar a los terapeutas en la elaboración de los libros de vida.




\section{Motivación}
Introducción al tema del TFG.


\section{Objetivos}
Este proyecto tiene como objetivo desarrollar un chatbot mediante el cual recabar información personal sobre la vida del paciente con demencia, clasificarla y estructurarla siguiendo un esquema que pueda facilitar la tarea de los terapeutas a la hora de construir un libro de vida. 

La clasificación de recuerdos se hará en base a unos criterios predefinidos por expertos en terapia ocupacional del proyecto CANTOR. Se clasificará en función de: 

\begin{itemize}
	\item \textbf{Emoción:} Se clasificarán los recuerdos en positivos y negativos, siendo estos últimos para identificar qué recuerdos no deben tratarse en las terapias por afligir al paciente. Los positivos se puntuarán de 0 a 10 en función de la felicidad que le traen al paciente. 
	\item \textbf{Etapa:} Los recuerdos pertenecerán a una de las siguientes etapas: infancia, adolescencia, edad adulta o tercera edad según el periodo temporal en que aconteció.
	\item \textbf{Categorías:} Cada recuerdo entrará dentro de una o varias categorías que recojan una característica del recuerdo. Ejemplos de categorías: guerra civil, bailes, ocio, familia, aficiones…
\end{itemize}

\section{Plan de trabajo}

El plan de trabajo ha consistido de tres etapas:
\begin{itemize}
	\item Investigación y construcción de prototipo en la que se ha consolidado la idea del TFG, investigado sobre demencia y terapia ocupacional basada en reminiscencia, elegido tecnologías y creado un prototipo de análisis de texto usando spaCy para identificar si un recuerdo es positivo o negativo.
	\item Programación del Chatbot y desarrollo de la memoria
	\item Pruebas, revisión de la memoria y entrega
\end{itemize}
