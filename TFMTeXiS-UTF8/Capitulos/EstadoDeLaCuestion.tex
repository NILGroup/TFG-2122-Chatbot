\chapter{Estado de la Cuestión}
\label{cap:estadoDeLaCuestion}

\section{Demencia}
La demencia \citep{demencia} es una condición neurodegenerativa con tendencia a crecer, caracterizada por un deterioro cognitivo que interfiere con la vida cotidiana afectando a la memoria, al pensamiento, al lenguaje, al juicio y al comportamiento. La demencia no es una enfermedad específica aunque la mayor parte de los casos de demencia son provocados por la enfermedad de Alzheimer. Muchas veces se confunde la demencia con una consecuencia más del envejecimiento cuando la senilidad no implica sufrir misma.

Hay muchos síntomas asociados a la demencia pero, en este trabajo, nos vamos a centrar en la pérdida de memoria. Trabajar los recuerdos de una persona que sufre demencia ayuda a retrasar los efectos de la demencia. Hablaremos para ello de la terapia ocupacional basada en reminiscencia.

\section{Terapia ocupacional}
La terapia ocupacional \citep{terapia} se centra en que el paciente sea capaz de participar en las actividades de la vida cotidiana. Es decir, se basa en ayudar al individuo a llevar una vida lo más normal posible adaptando las tareas cotidianas a realizar o el entorno para que pueda llevarlas a cabo.

\subsection{Terapia ocupacional basada en reminiscencia}
Sacar información del seminario cantor


\section{Proyecto CANTOR}
\citep{cantor}


\section{TFGs 2021}
Herramienta de ayuda guiada para la reminiscencia \citep{reminiscencia} : Generación de historias a partir de una base de conocimiento: recomendación de temas a tratar en la terapia + aplicación web que enlaza situaciones y vivencias mediante grafos y luego permite añadir recursos fotográficos asociado a un tema. (Más como un chatbot que va sugiriendo temas a tratar)

Sistema de asistencia para cuidados de enfermos del Alzheimer \citep{asistencia} : Página que guarda información sobre pacientes y terapeutas asociados. Información relevante + historia de vida formada por instancias de recuerdos

Extracción de preguntas a partir de imágenes para personas con problemas de memoria mediante técnicas de Deep Learning \citep{preguntas} : chat desplegado con telegram. De las fotos que tiene archivadas va preguntando al usuario cosas relacionadas con la imagen

Generación de resúmenes de video-entrevistas utilizando redes neuronales \citep{resumen} : transcripción de video-entrevistas a texto

Extracción de información personal a partir de redes sociales para la creación de un libro de vida \citep{rrss}

\section{Chatbots}


\subsection{Plataformas de desarrollo}


\section{PLN}
El procesamiento de lenguaje natural