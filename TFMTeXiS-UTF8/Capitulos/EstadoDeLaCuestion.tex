\chapter{Estado de la Cuestión}
\label{cap:estadoDeLaCuestion}

En este capítulo se introducen las diferentes temáticas relacionadas con el trabajo desarrollado. Se presenta el concepto de demencia y cómo ralentizar el deterioro de las personas que la padecen mediante terapias de reminiscencia. También se explican los diferentes trabajos previos que se han ido desarrollando en el marco del proyecto CANTOR y que están relacionados con este TFG. Entrando en detalles más técnicos, se explica la importancia del Procesamiento del Lenguaje Natural de cara al desarrollo de tecnologías que comprendan a los humanos y, también se da a conocer el contexto de los Chatbots y cómo implementarlos.


\section{Demencia}
La demencia \citep{demencia} es una condición neurodegenerativa progresiva, caracterizada por un deterioro cognitivo que interfiere con la vida cotidiana afectando a la memoria, al pensamiento, al lenguaje, al juicio y al comportamiento. La demencia no es una enfermedad específica aunque la mayor parte de los casos de demencia son provocados por la enfermedad de Alzheimer. Muchas veces se confunde la demencia con una consecuencia más del envejecimiento, cuando no tiene por qué ser así.

Hay muchos síntomas asociados a la demencia pero, en este trabajo, nos vamos a centrar en la pérdida de memoria. Actualmente no existen tratamientos que curen la enfermedad de Alzheimer, lo que hace necesario otro tipo de intervenciones para retrasar su desarrollo. Uno de los abordajes es el terapéutico, en concreto, trabajar los recuerdos de una persona que sufre demencia ayuda a retrasar los efectos de la misma. Hablaremos por ello de la terapia ocupacional basada en reminiscencia.

\section{Terapia ocupacional basada en reminiscencia}
La terapia ocupacional \citep{terocup} se centra en que el paciente sea capaz de participar en las actividades de la vida cotidiana. Es decir, se basa en ayudar al individuo a llevar una vida lo más normal posible adaptando las tareas cotidianas o el entorno para que pueda llevarlas a cabo.

La terapia ocupacional basada en reminiscencia se centra en mejorar la calidad de vida de la persona con demencia. Se trata de una técnica basada en la recuperación de recuerdos dentro de un periodo de tiempo en la vida de la persona con el objetivo de construir la historia de vida del sujeto. La historia de vida surge de la sucesión de acontecimientos que componen la totalidad de la vivencia del sujeto. Para la revisión del pasado de la persona con demencia, es fundamental introducir en la terapia estímulos que ayuden al paciente a recordar como por ejemplo, fotografías, objetos, música, etc. Los resultados de estas terapias están estudiados y concluyen que ayudan a ralentizar el deterioro cognitivo del usuario y mejoran su estado de ánimo.


\section{Trabajos previos}

Este trabajo está enmarcado en el proyecto CANTOR \citep{cantor} dedicado al desarrollo de herramientas digitales de apoyo para las terapias ocupacionales basadas en reminiscencia dirigidas a personas con demencia. El objetivo del proyecto es realizar un programa usando tecnologías de Inteligencia Artificial capaz de recopilar la historia de vida del paciente y capaz de presentar esa información conseguida para que pueda ser revisada posteriormente. 

Dentro de este contexto de trabajo, se han llevado a cabo diferentes proyectos por parte de antiguos alumnos de la Facultad de Informática enfocados a desarrollar distintas partes del proyecto CANTOR. Algunos de los relacionados con este TFG se cuentan a continuación: 

\begin{enumerate}
	\item Generación de historias a partir de una base de conocimiento de \cite{lcastilla}: Aplicación para usar en entrevistas con personas con demencia que sugiere temas de los que hablar y con los que poder recordar más fácilmente. Permite ir apuntando las ideas más importantes sobre cada tema tratado. La información recopilada se puede visualizar en forma de grafos o diccionarios porque la herramienta se encarga de ir enlazando los recuerdos que se van sacando. Además permite añadir, asociado a un tema, estímulos que ayuden a recordar al usuario, como recursos fotográficos. 
	\item Recuérdame, aplicación de apoyo para el tratamiento de personas con problema de memoria mediante terapias basadas en reminiscencia de \cite{cbarquilla}: Aplicación web dirigida a los terapeutas encargados de tratar a personas con demencia. Es una herramienta de ayuda a la hora de planificar las sesiones. Se usa también como almacén de recuerdos de los pacientes que luego también pueden consultar la historia de vida que se ha ido construyendo. 
	\item Generación de historias de vida usando técnicas de Deep Learning de \cite{calameda}: Desarrollo de un sistema capaz de redactar las historias de vida de las personas a partir de sus recuerdos estructurados. 
	\item Sistema de asistencia para cuidados de enfermos del Alzheimer de \cite{dbedinger}: Aplicación que guarda la información sobre los terapeutas y sus pacientes. Con ella se recogen los recuerdos del paciente. La información recopilada más relevante se utiliza para generar la historia de vida de la persona en forma de narración. 
	\item Extracción de información personal a partir de redes sociales para la creación de un libro de vida de \cite{paguilera}: Interfaz web donde se podrán consultar datos recopilados sobre personas con demencia en forma de historia de vida. Alimentando a la aplicación se encuentra un programa que se encarga de extraer la información del paciente de sus redes sociales de forma automática.
	\item Extracción de preguntas a partir de imágenes para personas con problemas de memoria mediante técnicas de Deep Learning de \cite{aaizel} : Se trata de una terapia en forma de chat con un bot de Telegram. El usuario se encarga de enviar fotos que puedan representar recuerdos suyos a la aplicación y la aplicación extrae de esas fotos preguntas relacionadas que puedan ayudar a la persona con demencia a recordar con más detalle lo relacionado con esa foto. 
	\item Generación de resúmenes de vídeo-entrevistas utilizando redes neuronales de \cite{dalcazar}: Aplicación web que utiliza redes neuronales para transcribir entrevistas en vídeo de personas con demencia y generar el resumen en español.
	\item Extracción de recuerdos de vídeos de entrevistas con personas con problemas de memoria de \cite{hgarcia}: Herramienta capaz de crear transcripciones automáticas de entrevistas a personas con demencia en vídeo. Además, extrae la información importante de forma estructurada y la presenta en forma de grafo. 
\end{enumerate}

\section{Procesamiento del lenguaje natural}

El Procesamiento del Lenguaje Natural \citep{pln} es un campo de la Inteligencia Artificial que estudia las interacciones entre personas y máquinas mediante el uso del lenguaje natural, es decir, investiga cómo pueden comunicarse las computadoras y los humanos de forma eficiente. El PLN es un campo que se ha estado desarrollando durante los últimos 50 años y que, aunque tuvo poco éxito en un principio, en la actualidad, se emplea en muchos ámbitos. Es por intereses económicos y prácticos que los desarrollos en este área de conocimiento se hayan realizado para las lenguas más habladas como el inglés, alemán, español y chino. No obstante, existen muchas herramientas muy potentes que trabajan muchos idiomas como por ejemplo, el traductor de Google. El desarrollo de técnicas de procesamiento de lenguaje natural es vital también para el funcionamiento de los chatbots (tema en el que está centrado este TFG) como podrían ser los tan reconocibles Siri de Apple y Alexa de Amazon. Es importante tener en cuenta que se avanza mucho más en el lenguaje por escrito ya que hay muchos más datos y es más fácil de guardar en formato electrónico.


\subsection{Modelos para el PLN}

Para que las máquinas puedan tratar el lenguaje natural es necesario describirlo en términos matemáticos porque los ordenadores solo entienden de bytes. Existen dos formas de modelar el lenguaje:

\begin{enumerate}
	\item Modelo gramatical: Esta formado por reglas de reconocimiento de patrones estructurales relacionados con la fonética, la morfología, la semántica, la sintaxis etc. Estas reglas las definen expertos lingüistas y son las que permiten a las máquinas reconocer lo que solicita la persona.
	\item Modelo probabilístico: Se aplican técnicas matemáticas para extraer el conocimiento. En vez de usar reglas gramaticales, se recoge una gran cantidad de ejemplos y datos para calcular la frecuencia de aparición de las diferentes unidades lingüísticas (letras, palabras, oraciones) y su probabilidad de aparecer en un contexto determinado. Con estas probabilidades se puede predecir mucha información. Este modelo es lo que se denomina aprendizaje automático.
\end{enumerate}

\subsection{Componentes del PLN}

Existen varios tipos de análisis para extraer información del lenguaje natural y sus usos dependerán del objetivo de la aplicación:

\begin{itemize}
	\item Análisis morfológico: Se analiza la estructura interna de las palabras para clasificarlas en categorías (sustantivos, verbos, adjetivos, etc.) y extraer los lemas 
	
	\item Análisis sintáctico: Consiste en estudiar la estructura sintáctica de las oraciones a partir de una gramática de la lengua en cuestión.
	
	\item Análisis semántico: Se utiliza para extraer el significado de las oraciones.
	
	\item Análisis pragmático: Añade al análisis el contexto de uso del lenguaje para mejorar la interpretación.
\end{itemize}



\subsection{Aplicaciones del procesamiento del lenguaje natural}

Algunas de las aplicaciones \citep{appspln} de mayor utilidad del PLN son:

\begin{enumerate}
	\item \textbf{Traducción automática de textos}: A día de hoy el nivel de traducción es bastante razonable. Existe la posibilidad de traducir textos largos a distintos idiomas. Es más, existen algunos navegadores que traducen automáticamente páginas web de un idioma a otro como por ejemplo ``Google Chrome''. Las traducciones suelen ser más fiables para ciertas parejas de idiomas. El inglés suele ser uno de los idiomas más desarrollados en este campo por ser de los más usados en el mundo. Por otro lado, aun no está muy conseguida la traducción de documentos complejos o de aquellos con muchos matices, ni siquiera de un idioma principal a otro. Para dichos textos seguirá siendo necesaria la intervención de un humano.
	\item \textbf{Sistemas conversacionales con PLN}: Son tecnologías de tipo Chatbot como Siri de Apple o el Asistente de Google. Son aplicaciones que entablan pequeñas conversaciones con el usuario y resuelven sus dudas, aunque a veces no con mucho éxito todavía. Se amplia la información en la sección \ref{chatbot}.
	\item \textbf{Respuestas automáticas a preguntas}: El primero que se hizo famoso fue el sistema Watson, desarrollado por IBM. Ganó algunos concursos televisivos contra humanos por su capacidad de responder todo tipo de preguntas gracias al amplio conocimiento que almacena en su base de datos.
	\item \textbf{Análisis de sentimiento} \citep{sentimentanalisis}: Se trata de analizar opiniones sobre personas, productos y temas varios. Su uso más destacable es en el ámbito de las redes sociales para analizar cómo interactúan los usuarios. Se amplia la información en la sección \ref{sentiment}.
	\item \textbf{Resúmenes de textos automáticos}: A día de hoy existe muchísima  información en Internet y muchos textos quedan sin leer por falta de tiempo. Los resúmenes automáticos ayudan a determinar si un texto merece ser leído al completo. 
	\item \textbf{Clasificación de documentos por categorías}: Ayuda a dirigir la información contenida en dichos documentos a los usuarios interesados, ahorrando tiempo. 
\end{enumerate}

\subsection{Ventajas del PLN}

\begin{itemize}
	\item El ahorro de tiempo en trabajos que antes se realizaban a mano.
	\item Ahorro de costes con procesos automáticas en vez de acudir a profesional.
	\item Agiliza el etiquetado manual de documentos.
	\item Ayuda a tomar decisiones de negocio. Por ejemplo: el análisis automático de redes sociales permite detectar una posible crisis de reputación con rapidez y atajarla con mayor brevedad.
	\item Facilita el turismo con las traducciones entre idiomas y mejora la comunicación entre personas de distintas culturas.
\end{itemize}

\subsection{Análisis de sentimiento} \label{sentiment}

El análisis de sentimiento \citep{sentimientoanalisis} es un método para identificar las emociones que se esconden tras un mensaje concreto y forma parte del procesamiento de lenguaje natural (PLN). Consiste en analizar las frases para extraer de ellas las opiniones o sentimientos acerca de un tema o producto.

Con este análisis \citep{sentanalisis} se pretende determinar quién es el sujeto del sentimiento, sobre qué o quién tiene ese sentimiento y categorizar esos sentimientos como positivos, negativos o neutros. Tras realizar el análisis del sentimiento se puede averiguar qué se esconde detrás de información subjetiva.

Una de las muchas aplicaciones de esta tecnología está enfocada al marketing, de forma que permite a las empresas averiguar qué es lo que quieren sus consumidores mediante el escrutinio de opiniones en redes sociales u otros medios.Estos sistemas tienen limitaciones, ya que no pueden detectar toda la complejidad del lenguaje humano. Se encuentran problemas a la hora de comprender el contexto en el que se encuentra un texto o para entender la ironía o el sarcasmo.

Algunas herramientas conocidas para el análisis del sentimiento son:
\begin{itemize}
	\item Lingmotif\footnote{https://ltl.uma.es} $\rightarrow$ Se trata de una herramienta de análisis de sentimiento desarrollada por la universidad de Málaga. Permite obtener valores precisos de las opiniones y sentimientos dentro de un texto.
	\item Opinion Finder\footnote{https://mpqa.cs.pitt.edu/opinionfinder/} $\rightarrow$ Sistema desarrollado por investigadores de la Universidad de Pittsburgh, Cornell y Utah. Permite identificar la subjetividad de frases y varios aspectos de la subjetividad dentro de las propias frases mediante el procesamiento de documentos. Funciona en Inglés.
	\item LIWC\footnote{https://www.liwc.app} $\rightarrow$ ``Linguistic Inquiry and Word Count'' es un programa capaz de analizar textos para calcular el uso que hacen las personas de distintas categorías de palabras. Permite saber si los emisores transmiten un mensaje con palabras positivas o negativas entre otras muchas opciones.
\end{itemize}


\section{Chatbots} \label{chatbot}

Un Chatbot o asistente virtual inteligente \citep{chatbot} es un programa informático capaz de mantener una conversación real con un usuario en lenguaje natural. Dan respuesta a dudas y tareas planteadas por los internautas. Para desarrollar un Chatbot se suele usar Procesamiento del Lenguaje Natural (PLN) y ``Machine Learning''.  Algunos ejemplos de asistentes virtuales serían los que recomiendan viajes y lugares turísticos, para la compra de billetes de avión, para aprender idiomas y mejorar las habilidades lingüísticas, gestiones en banca, para consultar dudas sobre contratos de telefonía móvil, etc.

Muchas empresas disponen de un Chatbot para atender a sus clientes y resolver las dudas más frecuentes. Así, de paso, ahorran costes y dejan las preguntas más difíciles para los ''call centres'' o los chats con agentes humanos. Se consigue  contratar a menos agentes, ya que muchas de las preguntas que realizan los usuarios son repetitivas y pueden ser respondidas por los bots. Son capaces de interpretar lo que el usuario pide a través del texto que introduce o lo que dice y de mantener una conversación y dar respuestas concretas. 

\subsection{Tipos}

Existen distintos tipos de Chatbots según la finalidad que tengan:
\begin{itemize}
	\item Asistentes que son capaces de mandar notificaciones simples como mensajes de texto o de Whatsapp. 
	\item Los que se han diseñado para escuchar al usuario en conversaciones cortas como Siri de Apple que interpreta lo que pide el usuario y resuelve dudas o tareas como el tiempo va a hacer, quién ha ganado un partido de fútbol o enviar un mensaje. Conocen las respuestas a las preguntas frecuentes.
	\item Aquellos que resuelven temas concretos relativamente complejos dentro de una misma temática como, por ejemplo, la compra de billetes de tren, la compra de entradas, resolver dudas sobre contratos y ofertas de telefonía móvil, etc.
	\item Existen algunos más complejos que mantienen conversaciones con los usuarios como si fueran personas con sentimientos, empatía, conocimiento, etc.
\end{itemize}

Todos estos asistentes virtuales necesitan entender el lenguaje natural para recibir las peticiones por un lado y por el otro ser capaces de generar lenguaje natural para contestar. 


\subsection{Herramientas de desarrollo}

Un bot conversacional se desarrolla dando forma a un programa que sea capaz de extraer los datos de las conversaciones, buscar en bases de datos la información necesaria y dar respuestas a los usuarios, entre otras cosas.Para distinguir las peticiones del usuario dentro del texto que introduce o dice, hay que entrenar al agente para que pueda aprender a reconocerlas. Para ello, se proporciona a la herramienta muchos ejemplos de frases que los clientes puedan llegar a decir. Cuando el usuario real introduzca una frase con una petición similar, el agente será capaz de deducir lo que pide. Para algunas de las peticiones, el Chatbot dará una respuesta inmediata y directa porque la solución es clara. En otros casos el agente requerirá más información del usuario y le planteará preguntas para recabar dicha información. Por ejemplo, en caso de querer hacer un pedido, el bot necesitará más información sobre lo que el cliente desea comprar y formulará la pregunta correspondiente para averiguarlo.

Normalmente, para asegurar que lo que el chatbot entiende se correponde con la solicitud del usuario, a cada petición detectada se le asigna un nivel de confianza según la probabilidad de que se haya detectado correctamente o no. Si la probabilidad calculada es baja se le hará otra pregunta al usuario para mejorar dicha confianza como por ejemplo, ``no te he entendido, ¿puedes repetir la pregunta?''. Se trata de intentar que el usuario formule la petición de otro modo que sea más comprensible para el Chatbot. El objetivo es afinar lo máximo posible las respuestas que proporciona evitando contestar algo incorrecto. Se debe entrenar mucho al asistente para llegar a un nivel de confianza óptimo (80\%, por ejemplo) y que dicho asistente sea eficaz y útil al cliente.

Algunas de las herramientas \citep{herrchatbot} que ayudan a desarrollar este tipo de programas inteligentes son:

\begin{itemize}
	\item Language Understanding (LUIS) de Microsoft: Servicio de procesamiento de lenguaje natural con acceso a una biblioteca de conocimiento que pueden utilizar los bots.
	\item Google Dialogflow: Es un servicio de Google Cloud con acceso a las técnicas de machine learning de Google. Funciona mediante la comprensión del lenguaje natural.
	\item Watson Assistant de IBM: Proporciona un servicio en la nube y funcionalidades como respuestas automáticas, procesamiento del lenguaje natural, reconocimiento de peticiones, etc.
	\item ChattyPeople: De las mejores plataformas. Permite crear bots en Facebook de forma personalizada.
	\item Botsify: Dotado de una inteligencia artificial muy refinada. Permite conectarse con muchas plataformas a través de plugins. 
	\item Telegram Bots: Pensado para crear chatbots desde la plataforma de mensajería de Telegram. También permite la integración con bots externos.
	\item Chatfuel: Pensada para cualquier usuario sin necesidad de conocimientos técnicos. Permite automatizar las respuestas a preguntas frecuentes y establecer reglas de mensajes automáticos.
	\item AIML: Es un lenguaje de marcado, es decir, formado por etiquetas o marcas como HTML. Se basa en la búsqueda de patrones para la creación de bots conversacionales llamados ``Alicebots''\footnote{https://github.com/ezabou/ElisaBot}. 
	\item Python\footnote{https://analyticsindiamag.com/top-python-libraries-for-chatbot-development/}: Creación de chatbots mediante librerías como spaCy\footnote{https://course.spacy.io/es}, Chatterbot o NLTK.
\end{itemize} 

