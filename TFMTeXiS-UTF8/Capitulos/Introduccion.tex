\chapter{Introducción}
\label{cap:introduccion}

\chapterquote{Conozca todas las teorías. Domine todas las técnicas, pero al tocar un alma humana sea apenas otra alma humana}{Carl Gustav Jung}


Las personas con Alzheimer u otros tipos de demencia pueden beneficiarse del uso de la llamada terapia basada en reminiscencia. Se basa en la construcción de un libro de vida del paciente que recopila recuerdos positivos de su vida. Esta información se puede utilizar posteriormente para ejercitar su memoria y retrasar el deterioro de la enfermedad. Además, permite aumentar el bienestar de los pacientes.

En el presente proyecto se propone el desarrollo de un chatbot que permita recopilar y estructurar esta información para ayudar a los terapeutas en la elaboración de los libros de vida.


\section{Motivación}
Introducción al tema del TFG.


\section{Objetivos}

Este proyecto tiene como objetivo desarrollar un chatbot con el que recabar información personal sobre la vida de la persona con demencia, clasificarla y estructurarla siguiendo un esquema que pueda facilitar la tarea de los terapeutas a la hora de construir un libro de vida. 

Este objetivo principal se divide en los siguientes objetivos específicos:
\begin{itemize}
	\item Creación de un chatbot que recoja los recuerdos de los usuarios
	\item Creación de una interfaz conversacional de un chatbot
	\item Clasificación de los recuerdos en base a unos criterios predefinidos por expertos en terapia ocupacional. Se clasificará en función de: 
	\begin{itemize}
		\item \textbf{Emoción:} Se clasificarán los recuerdos en positivos y negativos. Es importante identificar y evitar los recuerdos negativos en las terapias para no afligir al paciente. 
		\item \textbf{Etapa:} Los recuerdos pertenecerán a una de las siguientes etapas: infancia, juventud, edad adulta o tercera edad según el periodo temporal en que aconteció.
		\item \textbf{Categorías:} Cada recuerdo entrará dentro de una o varias categorías que recojan una característica del recuerdo. Ejemplos de categorías: guerra civil, bailes, ocio, familia, aficiones, amigos…
	\end{itemize}
	\item Almacenamiento de los recuerdos estructurados y clasificados en una base de datos 
	\item Desarrollo de una aplicación web con una interfaz gráfica atractiva y sencilla de usar
\end{itemize}


\section{Plan de trabajo}

Con el fin de cumplir con los objetivos planteados en la sección anterior, se ha fijado una planificación dividida en cuatro etapas:
\begin{itemize}
	\item Investigación sobre demencia y terapias ocupacionales basadas en reminiscencia. A nivel más técnico, investigación sobre herramientas existentes para el desarrollo de chatbots.
	\item Construcción de un prototipo inicial de análisis de textos y elección de tecnologías.
	\item Implementación del Chatbot y desarrollo de la memoria
	\item Pruebas, revisión de la memoria y entrega
\end{itemize}
