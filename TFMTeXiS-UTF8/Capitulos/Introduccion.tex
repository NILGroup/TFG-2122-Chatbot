\chapter{Introducción}
\label{cap:introduccion}

\chapterquote{Conozca todas las teorías, domine todas las técnicas, pero al tocar un alma humana sea simplemente otra alma humana}{Carl Gustav Jung}


Las personas con Alzheimer u otros tipos de demencia pueden beneficiarse del uso de la llamada terapia basada en reminiscencia. Se basa en la construcción de un libro de vida del paciente que recopila recuerdos positivos de su vida. Esta información se puede utilizar posteriormente para ejercitar su memoria y retrasar el deterioro de la enfermedad. Además, permite aumentar el bienestar de los pacientes.

En el presente proyecto se propone el desarrollo de un chatbot que permita recopilar y estructurar esta información para ayudar a los terapeutas en la elaboración de los libros de vida.


\section{Motivación}

Las terapias basadas en reminiscencia\footnote{https://orpea.es/terapia-de-reminiscencia-para-personas-con-demencia/} favorecen la evocación de recuerdos mediante la estimulación, comunicación y entrenamiento con el objetivo de preservar todo lo posible la reserva cognitiva y el sentido de identidad. Para incitar al paciente a rememorar ciertos recuerdos es muy importante que el terapeuta tenga en cuenta la historia de vida de esa persona y entender el contexto en el que cada recuerdo se enmarca. Un terapeuta no puede trabajar un recuerdo con el usuario sin tener en cuenta las consecuencias que puedan suponer para esa persona, no solo por el hecho de hacer referencia a él sino también la forma de tratarlo. El paciente puede reaccionar de muchas formas pudiendo llegar a entristecer, enfadarse, etc. agravando así su situación.

Sabiendo la importancia que tiene la historia de vida en este tipo de terapias, los terapeutas se encargan de narrarla, proceso que requiere mucho trabajo y que suelen hacer manualmente. Esto implica recoger los datos de la persona para luego transformarlos en un relato biográfico. Poder acelerar este proceso mediante la recopilación automática de esa información ahorra mucho tiempo y esfuerzo a los terapeutas que pueden dedicarlo al trabajo con el paciente para ralentizar su deterioro.


\section{Objetivos}

Este proyecto tiene como objetivo desarrollar un chatbot con el que recabar información personal sobre la vida de la persona con demencia, clasificarla y estructurarla siguiendo un esquema que pueda facilitar la tarea de los terapeutas a la hora de construir un libro de vida. 

Este objetivo principal se divide en los siguientes objetivos específicos:
\begin{itemize}
	\item Creación de un chatbot que recoja los recuerdos de los usuarios.
	\item Creación de una interfaz conversacional de un chatbot.
	\item Clasificación de los recuerdos en base a unos criterios predefinidos por expertos en terapia ocupacional. Se categorizarán en función de: 
	\begin{itemize}
		\item \textbf{Emoción:} Se clasificarán los recuerdos en positivos y negativos. Es importante identificar y evitar los recuerdos negativos en las terapias para no afligir al paciente. 
		\item \textbf{Etapa:} Los recuerdos pertenecerán a una de las siguientes etapas: infancia, juventud, edad adulta o tercera edad según el periodo temporal en que acontecieron.
		\item \textbf{Categorías:} Cada recuerdo entrará dentro de una o varias categorías que recojan una característica del recuerdo. Ejemplos de categorías: guerra civil, bailes, ocio, familia, aficiones, amigos…
	\end{itemize}
	\item Almacenamiento de los recuerdos estructurados y clasificados en una base de datos .
	\item Desarrollo de una aplicación web que permita al usuario interactuar con el chatbot a través de una interfaz gráfica atractiva y sencilla de usar.
\end{itemize}


\section{Plan de trabajo}

Con el fin de cumplir con los objetivos planteados en la sección anterior, se ha fijado una planificación dividida en cuatro etapas:
\begin{itemize}
	\item Investigación sobre demencia y terapias ocupacionales basadas en reminiscencia. A nivel más técnico, investigación sobre herramientas existentes para el desarrollo de chatbots.
	\item Construcción de un prototipo inicial de análisis de textos y elección de tecnologías.
	\item Implementación del Chatbot y desarrollo de la memoria del TFG.
	\item Pruebas, revisión de esta memoria y entrega.	
\end{itemize}

\section{Estructura de la memoria}

La presente memoria está compuesta por cinco capítulos, entre ellos esta introducción. Cada uno de estos capítulos se expone a continuación: 

\begin{itemize}
	\item \textbf{Capítulo 2:} En el estado de la cuestión se presenta el concepto de demencia y de las terapias ocupacionales basadas en reminiscencia. También se explican los diferentes trabajos previos que se han ido desarrollando en el marco del proyecto CANTOR. Se da a conocer la importancia del procesamiento del lenguaje natural y el contexto de los chatbots y cómo implementarlos.
	\item \textbf{Capítulo 3:} El apartado de la arquitectura del chatbot está centrado en cómo se ha desarrollado el bot conversacional de forma que pueda recopilar los recuerdos del usuario con demencia y almacenarlos de manera estructurada. Para ello, se explican los componentes del chatbot que son los siguientes: el módulo de preguntas, el módulo de clasificación de recuerdos y el módulo de elección de la mejor siguiente pregunta. 
	\item \textbf{Capítulo 4:} En el capítulo de la aplicación web se explican los pasos que se han seguido para desarrollar la página web desde el principio con un prototipo inicial hasta el final con una aplicación web funcional y con distintas vistas. Entre medias se describe el uso de una base de datos MySQL.
	\item \textbf{Capítulo 5:} Se trata del capítulo de conclusiones y trabajo futuro en el que se recogen las conclusiones extraídas de los resultados obtenidos en este trabajo. También se detallan las posibles líneas de trabajo que deja abiertas este TFG para su continuación en un futuro.
\end{itemize}

