\chapter{Introduction}
\label{cap:introduction}

\chapterquote{Know all the theories, master all the techniques, but as you touch a human soul be just another human soul}{Carl Gustav Jung}

People who suffer Alzheimer's or other types of dementia may benefit from the use of the  so-called reminiscence-based therapy. It is based on the composition of a patient's life book that gathers positive memories from his or her life. This information can then be used to exercise their memory and delay the damage that causes the disease. It also helps increasing the patient's well-being.

In this project the main goal is to develop a chatbot to collect and structure this information. It helps the therapists in the elaboration of life books.


\section{Motivation}

Reminiscence-based therapies\footnote{https://orpea.es/terapia-de-reminiscencia-para-personas-con-demencia/} encourage the evocation of memories through stimulation, communication and training with the aim of preserving as much cognitive reserve and sense of identity as possible. In order to prompt the patient to recall certain memories, it is very important for the therapist to consider the person's life history and to understand the context in which each memory is framed. A therapist cannot work on a memory with the user without taking into account the consequences for that person, not only by referring to it but also by the way it is dealt with. The patient may react in many ways and may become sad, angry, etc., thus aggravating his/her situation.

Knowing the importance of the life history in this type of therapy, therapists are in charge of narrating it, a process that requires a lot of work and which is usually done manually. This involves collecting the person's data and then transforming it into a biographical account. Being able to speed up this process by automatically collecting this information saves therapists a lot of time and effort that can better be spent on working with the patient to slow down their deterioration.

\section{Objectives}

This project aims to develop a chatbot with which to collect personal information about the life of the person with dementia, classify it and structure it according to a scheme that can facilitate the task of therapists when building a life book.

This main objective is divided into the following specific goals:
\begin{itemize}
	\item Creation of a chatbot that collects users' memories.
	\item Creation of a chatbot conversational interface.
	\item Classification of memories based on criteria predefined by occupational therapy experts. They will be classified according to:
	\begin{itemize}
		\item \textbf{Emotion} Memories shall be categorised into positive and negative. It is important to identify and avoid negative memories in therapy so as not to distress the patient.
		\item \textbf{Stage:} The memories will belong to one of the following stages: childhood, youth, adulthood or old age depending on the time period in which they occurred.
		\item \textbf{Categories:} Each memory will fall into one or more categories that capture a characteristic of the memory. Examples of categories: civil war, dance, leisure, family, hobbies, friends...
	\end{itemize}
	\item Storage of memories structured and classified in a database.
	\item Development of a web application that allows the user to interact with the chatbot through an attractive and easy-to-use graphical interface.
\end{itemize}

\section{Work plan}

In order to meet the objectives set out in the previous section, a plan divided into four stages has been established:
\begin{itemize}
	\item Research on dementia and reminiscence-based occupational therapies. On a more technical level, research on existing tools for the development of chatbots.
	\item Construction of an initial prototype for text analysis and choice of technologies.
	\item Chatbot implementation and development of this document.
	\item Testing, document's review and turn in the project.
\end{itemize}

\section{Document structure}

This document is composed of five chapters, including this introduction. Each of these chapters is presented below: 

\begin{itemize}
	\item \textbf{Chapter 2:} The state of the art presents the concept of dementia and reminiscence-based occupational therapies. It also explains the different previous projects that have been developed in the context of the CANTOR project. Furthermore, it illustrates the importance of natural language processing and the context of chatbots and how to implement them.
	\item \textbf{Chapter 3:} The section on the architecture of the chatbot focuses on how the conversational bot has been developed in such a way that it can collect memories of the user with dementia and store them in a structured manner. For this purpose, the components of the chatbot are explained in this chapter. They are the following: the question's module, the memories sorting module and the module for choosing the next best question. 
	\item \textbf{Chapter 4:} The web application chapter explains the steps that have been followed to develop the website from the beginning with an initial prototype to the end with a functional web application with different views. In between, the use of a MySQL database is described.
	\item \textbf{Chapter 5:} This is the conclusions and future work chapter, which includes the conclusions drawn from the results obtained in this work. It also details the possible lines of work that this TFG leaves open for its continuation in the future.
\end{itemize}







