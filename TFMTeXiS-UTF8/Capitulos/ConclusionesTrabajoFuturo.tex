\chapter{Conclusiones y Trabajo Futuro}
\label{cap:conclusiones}

El objetivo de este TFG era construir un Chatbot lo suficientemente inteligente como para conseguir recopilar información sobre la vida de una persona con demencia y almacenarla de forma estructurada. Aunque todavía se puede realizar mucho trabajo a partir de lo ya desarrollado, se ha conseguido construir una aplicación funcional que sirve como herramienta de apoyo, relativamente autónoma, en la dura tarea de redactar una historia de vida. 

En un principio se había pensado desarrollar un bot conversacional con el que poder mantener una conversación sofisticada y coherente. Con el paso del tiempo, al encontrarse con numerosas trabas y viendo la complejidad que suponía simular una entrevista del terapeuta, se fue distorsionando un poco la idea inicial. Se ha terminado creado una herramienta bastante útil porque cumple con la función de recoger la información que proporciona el usuario aunque no tan inteligente como se proponía en un principio, sobre todo a la hora de darle humanidad. Escasea la capacidad de ayudar a la persona con demencia a recordar mediante estímulos y a través de una conversación fluida. 

En cualquier caso, se ha conseguido realizar un análisis bastante preciso de las respuestas que va dando el usuario a través del chat. Se logra estructurar los recuerdos en las etapas de la vida de la persona a las que corresponden de forma bastante exacta al igual que ocurre con el análisis de sentimiento de los textos. Aunque podrían hacer una distinción más concreta, sobre todo en el caso de recuerdos que no entren en ninguna de las categorías proporcionadas, se deja como posible trabajo futuro. Esto es porque para las tecnologías que se han utilizado, el volumen de datos era bastante pequeño, hubiese funcionado mejor con un dataset mucho más grande de recuerdos. Por otro lado, la conversación se hace más fluida gracias al módulo de encadenamiento de respuestas y preguntas que analiza la contestación del usuario para encontrar la mejor siguiente pregunta a plantear. Con un mayor análisis del contexto de la frase hubiese mejorado mucho la elección de preguntas coherentes y fluidas.

A nivel aplicación web, es bastante manejable e interactiva, ofreciendo una interfaz sencilla de utilizar y  dirigida tanto a pacientes como a terapeutas. Ofrece un control sobre los usuarios y sobre la información que se va recopilando. No se han añadido más funcionalidades porque el principal foco debía ser el Chatbot y, además, ya existían otros TFGs que se centraban en un desarrollo web más específico. 

En lo relativo al trabajo futuro, se proponen las siguientes mejoras:
\begin{itemize}
	\item Un análisis de sentimiento y clasificación en etapas de vida más exacto y concreto. Una mejora de este análisis utilizando un dataset de recuerdos mucho más grande para el entrenamiento del modelo. 
	\item Una conversación más fluida con el usuario. Aparte de preguntar al usuario sobre su pasado, implementar un módulo conversador que simplemente reaccione a las respuestas, siempre acorde con lo que ha dicho la persona. También incluir un programa para completar la información que falte por conocer del usuario según una plantilla a rellenar.
	\item Añadir alguna funcionalidad extra a la aplicación web como definir terapias concretas dirigidas a una temática específica, permitir al terapeuta añadir preguntas a la batería disponible en la base de datos, habilitar la opción de reconocimiento de voz para que al paciente con demencia le resulte mucho más fácil la interacción con el Chatbot, etc.
	\item Una mejor elección de las cuestiones que se le planteen al usuario. Aparte de comparar la respuesta dada por el usuario para elegir la siguiente pregunta, también comparar con la cuestión que se hizo a la que el usuario contestó. Por ejemplo, si se ha preguntado por algo relacionado con la infancia, que se siga preguntando por cosas relacionadas con la infancia. También se pueden implementar ontologías para que al analizar las respuestas se puedan relacionar palabras o conceptos con posibles preguntas a formular.
	\item Ampliar la batería de preguntas y de temas a tratar mediante la adaptación de entrevistas tipo que realizan los terapeutas ocupacionales.
	\item Crear la posibilidad de que las preguntas se puedan generar solas, sin estar previamente redactadas, según la temática que se está tratando y los datos que va proporcionando el paciente. 
	\item Mayor categorización de los recuerdos con etiquetas como familia, amigos, ocio, comida, vacaciones, aficiones, etc.r 
\end{itemize}