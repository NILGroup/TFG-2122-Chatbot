\chapter{Arquitectura}
\label{cap:arquitectura}

En el estado de la cuestión es donde aparecen gran parte de las referencias bibliográficas del trabajo. Una de las formas más cómodas de gestionar la bibliografía en {\LaTeX} es utilizando \textbf{bibtex}. Las entradas bibliográficas deben estar en un fichero con extensión \textit{.bib} (con esta plantilla se proporcionan 3, dos de los cuales están vacíos). Cada entrada bibliográfica tiene una clave que permite referenciarla desde cualquier parte del texto con los siguiente comandos:

\begin{itemize}
	\item Referencia bibliografica con cite: \cite{ldesc2e}
	\item Referencia bibliográfica con citep: \citep{notsoshort}
	\item Referencia bibliográfica con citet: \citet{latexAPrimer}
\end{itemize}

Es posible citar más de una fuente, como por ejemplo \citep{latexCompanion,LaTeXLamport,texKnuth}

