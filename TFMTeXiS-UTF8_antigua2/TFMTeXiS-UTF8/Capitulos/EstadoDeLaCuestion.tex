\chapter{Estado de la Cuestión}
\label{cap:estadoDeLaCuestion}

Introducción de lo que voy a hablar en el estado de la cuestión y por qué

\section{Demencia}
La demencia \footnote{https://www.alz.org/alzheimer-demencia/que-es-la-demencia?lang=es-MX} es una condición neurodegenerativa progresiva, caracterizada por un deterioro cognitivo que interfiere con la vida cotidiana afectando a la memoria, al pensamiento, al lenguaje, al juicio y al comportamiento. La demencia no es una enfermedad específica aunque la mayor parte de los casos de demencia son provocados por la enfermedad de Alzheimer. Muchas veces se confunde la demencia con una consecuencia más del envejecimiento, cuando no tiene por qué ser así.

Hay muchos síntomas asociados a la demencia pero, en este trabajo, nos vamos a centrar en la pérdida de memoria. Trabajar los recuerdos de una persona que sufre demencia ayuda a retrasar los efectos de la misma. Hablaremos para ello de la terapia ocupacional basada en reminiscencia.

\section{Terapia ocupacional basada en reminiscencia}
La terapia ocupacional \footnote{https://aptoca.org/terapia-ocupacional/que-es-la-terapia-ocupacional-2/} se centra en que el paciente sea capaz de participar en las actividades de la vida cotidiana. Es decir, se basa en ayudar al individuo a llevar una vida lo más normal posible adaptando las tareas cotidianas a realizar o el entorno para que pueda llevarlas a cabo.

(Sacar información del seminario cantor)
La terapia ocupacional basada en reminiscencia se centra en mejorar la calidad de vida de la persona con demencia. Se trata de una técnica basada en la recuperación de recuerdos dentro de un periodo de tiempo en la vida de la persona con el objetivo de construir la historia de vida del sujeto. La historia de vida surge de la sucesión de acontecimientos que componen la totalidad de la vivencia del sujeto.


%\section{TFGs 2021}
\section{Trabajo previo}
Herramienta de ayuda guiada para la reminiscencia \citep{reminiscencia} : Generación de historias a partir de una base de conocimiento: recomendación de temas a tratar en la terapia + aplicación web que enlaza situaciones y vivencias mediante grafos y luego permite añadir recursos fotográficos asociado a un tema. (Más como un chatbot que va sugiriendo temas a tratar)

Sistema de asistencia para cuidados de enfermos del Alzheimer \citep{asistencia} : Página que guarda información sobre pacientes y terapeutas asociados. Información relevante + historia de vida formada por instancias de recuerdos

Extracción de preguntas a partir de imágenes para personas con problemas de memoria mediante técnicas de Deep Learning \citep{preguntas} : chat desplegado con telegram. De las fotos que tiene archivadas va preguntando al usuario cosas relacionadas con la imagen

Generación de resúmenes de video-entrevistas utilizando redes neuronales \citep{resumen} : transcripción de video-entrevistas a texto

Extracción de información personal a partir de redes sociales para la creación de un libro de vida \citep{rrss}

Alameda Salas, María Cristina (2022) Generación de historias de vida usando técnicas de Deep Learning.

Barquilla Blanco, Cristina y Díez García, Patricia y Mulas López, Santiago Marco y Verdú Rodríguez, Eva (2022) Recuérdame: Aplicación de apoyo para el tratamiento de personas con problema de memoria mediante terapias basadas en reminiscencia. 

García González, Hugo (2022) Extracción de recuerdos de vídeos de entrevistas con personas con problemas de memoria. 

\section{PLN}
El procesamiento de lenguaje natural

\subsection{Chatbots}

\subsection{Análisis de sentimiento}

El análisis de sentimiento \footnote{https://gaeapeople.com/marketing-estrategia/sentiment-analysis} es un método para identificar las emociones que se esconden tras un mensaje concreto y forma parte del procesamiento de lenguaje natural (PLN). Consiste en analizar las frases para extraer de ellas las opiniones o sentimientos acerca de un tema o producto.

Con este análisis \footnote{https://blog.pangeanic.es/funcionamiento-herramientas-analisis-sentimiento-basadas-en-inteligencia-artificial} se pretende determinar quién es el sujeto del sentimiento, sobre qué o quién tiene ese sentimiento y categorizar esos sentimientos como positivos, negativos o neutros.

Tras realizar el análisis del sentimiento se puede averiguar qué se esconde detrás de información subjetiva.

Una de las muchas aplicaciones de esta tecnología está enfocada al marketing, de forma que permite a las empresas averiguar qué es lo que quieren sus consumidores mediante el escrutinio de opiniones en redes sociales u otros medios.

Estos sistemas tienen limitaciones, ya que no pueden detectar toda la complejidad del lenguaje humano. Se encuentran problemas a la hora de comprender el contexto en el que se encuentra un texto o para entender la ironía o el sarcasmo.

Las principales herramientas para el análisis del sentimiento son:
\begin{itemize}
	\item Lingmotif \footnote{https://ltl.uma.es} $\rightarrow$ Se trata de una herramienta de análisis de sentimiento desarrollada por la universidad de Málaga. Permite obtener valores precisos de las opiniones y sentimientos dentro de un texto.
	\item Opinion Finder \footnote{https://mpqa.cs.pitt.edu/opinionfinder/} $\rightarrow$ Sistema desarrollado por investigadores de la Universidad de Pittsburgh, Cornell y Utah. Permite identificar la subjetividad de frases y varios aspectos de la subjetividad dentro de las propias frases mediante el procesamiento de documentos. Funciona en Inglés.
	\item LIWC \footnote{https://www.liwc.app} $\rightarrow$ ``Linguistic Inquiry and Word Count'' es un programa capaz de analizar textos para calcular el uso que hacen las personas de distintas categorías de palabras. Permite saber si los emisores transmiten un mensaje con palabras positivas o negativas entre otras muchas opciones.
\end{itemize}


